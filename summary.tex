\documentclass[12pt,letterpaper]{article}

\usepackage[left=2.5cm,top=2cm,right=2.5cm,bottom=2.5cm,nohead,nofoot]{geometry}
\usepackage{setspace}

%\pagestyle{empty}

\begin{document}

\begin{flushright}
Tony Miller\\
Nik Nyby\\
Matt Hurtado
\end{flushright}

\begin{center}
\section*{Pineapple Tracker project summary}
\end{center}

% outline:
% ~~~~~~~~
% part 1: intro -- what is this project about?
% part 2: audio engine
% part 3: UI
% part 4: other things we learned
% part 5: plans for the future
% part 6:

\doublespacing
\par
Our aim for this project was to extend a music tracker's user interface to behave more like Vi, and to learn a little bit about sound synthesis. A tracker is a type of computer music composition tool. Many computer music tools give you some sort of graphical interface for entering notes and commands, usually resembling piano keys. Trackers give you a column in which you have to enter all notes and commands, often in hexadecimal. It is a minimal and efficient concept. Vi is a text editor that makes an effort to be efficient to use, and friendly to everyday users. The commands are obscure, for example the h, j, k, and l keys move the cursor, instead of the expected arrow keys. Vi has a learning curve that is initially steep but 

\par
We wanted to control our tracker just like we controlled Vi. We were surprised that there isn't already a tracker that does this, because a tracker's interface can really benefit from being fast and efficient to use. Our other goals included rewriting the audio engine, adding more functionality such as filtering, granular synthesis, and a general effort to try to make our code understandable and orthogonal.

\par
We started the quarter not knowing much about digital sound processing, so we didn't do anything to the audio engine for the first five weeks. We decided we wanted to be able to change the sampling rate, so we ended up creating a different frequency table to keep the tracker in tune with the new sampling rate. We found out this wasn't the best solution because certain effects like vibrato and note-bend were still affected by the frequency changed. We learned from our contract sponsor that we will need to resample our output wave if we want to arbitrarily change the sample rate.

\par
The work we put into the user interface was probably the most fruitful, as it led us closest to our goal. This is due to a couple of things. I think the most important is that we knew exactly what we wanted before we started coding. Because we already know how to use Vi, we sort of got a specification for free since we knew exactly how we wanted the commands to behave. Since we didn't do much planning, this allowed our work on the user interface to be successful. We did make our own specification about halfway through the quarter.

\par
When we began working on this project we had lots of ideas for features, but since we lack experience it was hard to gauge which ones would be more challenging than others. Tempo control turned out to be a feature that was easier to implement than we had expected, while others that seem trivial (like JACK support) aren't done yet. This made it hard to plan before-hand, and sometimes it wasn't clear what to focus on.

\par
There are some things we want to work on in the future to make Pineapple Tracker more useful. One is the ability to import MOD files. We also have a goal of rewriting the audio engine for educational purposes.

\end{document}
